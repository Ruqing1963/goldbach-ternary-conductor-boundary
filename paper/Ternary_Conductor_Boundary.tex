\documentclass[11pt,a4paper]{article}

\usepackage[utf8]{inputenc}
\usepackage[T1]{fontenc}
\usepackage{amsmath,amssymb,amsthm,mathrsfs}
\usepackage{geometry}
\geometry{margin=1in}
\usepackage{xcolor}
\usepackage[breaklinks=true,colorlinks=true,linkcolor=blue!60!black,citecolor=blue!60!black,urlcolor=blue!50!black]{hyperref}
\usepackage{booktabs}
\usepackage{array}
\usepackage{graphicx}
\usepackage{float}

\newtheorem{theorem}{Theorem}[section]
\newtheorem{lemma}[theorem]{Lemma}
\newtheorem{proposition}[theorem]{Proposition}
\newtheorem{corollary}[theorem]{Corollary}
\theoremstyle{definition}
\newtheorem{definition}[theorem]{Definition}
\newtheorem{conjecture}[theorem]{Conjecture}
\newtheorem{remark}[theorem]{Remark}
\newtheorem{observation}[theorem]{Observation}

\DeclareMathOperator{\ord}{ord}
\DeclareMathOperator{\GSp}{GSp}
\DeclareMathOperator{\Jac}{Jac}
\DeclareMathOperator{\rad}{rad}

% ═══════════════════════════════════════════════════════════════════════════════
\title{The Ternary Conductor Boundary:\\
Why Conductor Rigidity Is Specific to the Binary Goldbach Problem}
\author{Ruqing Chen\\[4pt]
\textit{GUT Geoservice Inc., Montr\'{e}al, QC, Canada}\\[2pt]
\texttt{ruqing@hotmail.com}}
\date{February 2026}

\begin{document}
\maketitle

% ═══════════════════════════════════════════════════════════════════════════════
\begin{abstract}
We investigate whether the conductor rigidity framework established for the
binary Goldbach conjecture ($p + q = N$, genus~$2$, $\GSp(4)$) extends to
the ternary Goldbach problem ($p_1 + p_2 + p_3 = N$, genus~$3$, $\GSp(6)$).
By computing the \emph{true} discriminant of the genus-3 hyperelliptic Frey
curve $C : y^2 = x(x^2 - p_1^2)(x^2 - p_2^2)(x^2 - p_3^2)$, we show that
the algebraic identity $p^2 - q^2 = (p - q)N$ which underpins binary conductor
rigidity has \emph{no} ternary analogue: the integer $N$ does not appear as an
independent factor in the genus-3 discriminant.  Consequently, the Band
Shifting Law ceases to hold ($R^2 = 0.0002$ against the static conduit
variable~$\xi$).  Instead, the ternary conductor decomposes into three
geometric components---summand, difference, and partial-sum
contributions---whose interplay is governed by prime-factor statistics rather
than algebraic structure.  The PPP--CCC gap shrinks from a large, stable
separation (binary) to a small, fluctuating offset of $0.10 \pm 0.07$
(ternary).  These results precisely delineate the \emph{applicability boundary}
of conductor rigidity: the theory is native to genus~$2$, where the
factorisation $p^2 - q^2 = (p-q)N$ embeds $N$ as a geometric invariant of
the Frey family.
\end{abstract}


% ═══════════════════════════════════════════════════════════════════════════════
\section{Introduction}\label{sec:intro}

The companion papers~\cite{Chen2026DS,Chen2026AV,Chen2026GMII} established a
conductor rigidity framework for the binary Goldbach conjecture.  The central
tool is the genus-2 Frey curve $C_{\mathrm{bin}} : y^2 = x(x^2 - p^2)(x^2 - q^2)$,
whose discriminant factors as
\begin{equation}\label{eq:binfactor}
  \Delta_{\mathrm{bin}} \;\propto\;
    (pq)^6 \cdot (p^2 - q^2)^4
    \;=\; (pq)^6 \cdot (p-q)^4 \cdot N^4,
\end{equation}
since $p + q = N$.  The appearance of $N^4$ as an \emph{independent algebraic
factor} is the origin of conductor rigidity: it creates a deterministic
``static conduit'' $\rad_{\mathrm{odd}}(N/2)$ that shifts the entire conductor
band linearly, yielding the Band Shifting Law with $R^2 > 0.997$.

A natural question is whether this framework extends to the ternary Goldbach
problem $N = p_1 + p_2 + p_3$.  The corresponding genus-3 Frey curve is
\begin{equation}\label{eq:terncurve}
  C_{\mathrm{tern}} : y^2 = x(x^2 - p_1^2)(x^2 - p_2^2)(x^2 - p_3^2).
\end{equation}
In this paper, we compute the true discriminant of $C_{\mathrm{tern}}$ and
show that $N$ does \emph{not} factor out as an independent term.  This
algebraic obstruction has profound consequences for the conductor
landscape.\footnote{Scripts and data:
\url{https://github.com/Ruqing1963/goldbach-ternary-conductor-boundary}.}


% ═══════════════════════════════════════════════════════════════════════════════
\section{The True Discriminant at Genus~3}\label{sec:disc}

\subsection{Discriminant computation}

The curve~\eqref{eq:terncurve} has non-zero roots $\pm p_1, \pm p_2, \pm p_3$
and a root at the origin.  The discriminant of a hyperelliptic curve
$y^2 = f(x)$ is proportional to the product of squared differences of all
roots of~$f$.  For our degree-7 polynomial, this gives:

\begin{proposition}[Genus-3 discriminant]\label{prop:disc}
  \begin{equation}\label{eq:disc}
    \Delta_{\mathrm{tern}} \;\propto\;
    \prod_{i=1}^{3} p_i^{6}
    \;\cdot\; \prod_{i < j} (p_i^2 - p_j^2)^4.
  \end{equation}
\end{proposition}

\begin{proof}
  The seven roots of $f(x) = x(x^2 - p_1^2)(x^2 - p_2^2)(x^2 - p_3^2)$ are
  $0, \pm p_1, \pm p_2, \pm p_3$.  The factors involving the root at~$0$
  contribute $\prod (\pm p_i)^2 = \prod p_i^2$, raised to the fourth power by
  the discriminant formula, giving $\prod p_i^{6}$ after accounting for
  multiplicity.  The remaining factors are the squared differences
  $((\pm p_i)^2 - (\pm p_j)^2)$ between distinct non-zero roots, which
  reduce to $(p_i^2 - p_j^2)^4$ for each pair~$\{i,j\}$.
\end{proof}

\subsection{Factorisation under the ternary constraint}

Using $p_1 + p_2 + p_3 = N$, each squared difference factors as:
\begin{equation}\label{eq:factor}
  p_i^2 - p_j^2 = (p_i - p_j)(p_i + p_j) = (p_i - p_j)(N - p_k),
\end{equation}
where $\{i,j,k\}$ is a permutation of $\{1,2,3\}$.  Thus:

\begin{corollary}[Expanded discriminant]\label{cor:expanded}
  \begin{equation}\label{eq:expanded}
    \Delta_{\mathrm{tern}} \;\propto\;
    \prod_{i=1}^{3} p_i^{6}
    \;\cdot\; \prod_{i < j} (p_i - p_j)^{4}
    \;\cdot\; \prod_{k=1}^{3} (N - p_k)^{4}.
  \end{equation}
\end{corollary}

\begin{remark}[The critical difference from genus~2]\label{rem:critical}
  In the binary discriminant~\eqref{eq:binfactor}, $N$ appears as an
  \emph{independent factor} because $p + q = N$ implies
  $p^2 - q^2 = (p-q) \cdot N$.  In the ternary
  discriminant~\eqref{eq:expanded}, $N$ enters \emph{only} through the
  partial sums $(N - p_k) = p_i + p_j$.  There is no independent $N$-factor
  that can be extracted as a universal ``static conduit.''

  Concretely: if an odd prime $r \mid N$ but $r \nmid p_k$ and
  $r \nmid (p_i - p_j)$ for all $\{i,j,k\}$, then $r \nmid \Delta_{\mathrm{tern}}$
  and the curve has good reduction at~$r$.  This cannot happen in genus~2,
  where $r \mid N$ forces $r \mid \Delta_{\mathrm{bin}}$.
\end{remark}


% ═══════════════════════════════════════════════════════════════════════════════
\section{The Corrected Conductor Proxy}\label{sec:proxy}

\begin{definition}[True ternary conductor proxy]\label{def:proxy}
  The conductor proxy is
  \[
    \rho_3 \;=\; \frac{\log\bigl(\rad_{\mathrm{odd}}(\Delta_{\mathrm{tern}})\bigr)}
                      {\log N},
  \]
  where $\rad_{\mathrm{odd}}$ denotes the product of all odd primes
  dividing~$\Delta_{\mathrm{tern}}$.  Since $\rad$ strips exponents:
  \begin{equation}\label{eq:rad}
    \rad_{\mathrm{odd}}(\Delta_{\mathrm{tern}}) =
    \rad_{\mathrm{odd}}\!\bigl(
      p_1  p_2  p_3
      \cdot \textstyle\prod_{i<j}(p_i - p_j)
      \cdot \textstyle\prod_{k}(N - p_k)
    \bigr).
  \end{equation}
\end{definition}

This proxy admits a natural \emph{geometric decomposition}:
\begin{equation}\label{eq:decomp}
  \rho_3 = \sigma + \delta + \pi,
\end{equation}
where:
\begin{itemize}
\item $\sigma = \log(\rad_{\mathrm{odd}}(p_1 p_2 p_3))/\log N$ is the
  \textbf{summand contribution}: the odd primes from the summands themselves.
\item $\delta$ is the \textbf{difference contribution}: the log-sum of
  \emph{new} odd primes from $(p_i - p_j)$ not already counted in~$\sigma$.
\item $\pi$ is the \textbf{partial-sum contribution}: the log-sum of
  \emph{new} odd primes from $(N - p_k)$ not already counted in~$\sigma$
  or~$\delta$.
\end{itemize}


% ═══════════════════════════════════════════════════════════════════════════════
\section{Computational Results}\label{sec:results}

We scan all unordered triples $(p_1, p_2, p_3)$ with $p_1 \leq p_2 \leq p_3$
and $p_1 + p_2 + p_3 = N$ for odd $N \in [501, 4001]$.  Each triple is
classified as PPP (all prime), CCC (all composite), or mixed.

\subsection{Decomposition at $N = 1025$}

\begin{table}[H]
\centering
\begin{tabular}{l r r r r}
\toprule
Category & $\langle\rho_3\rangle$ & $\sigma$ (summands) &
  $\delta$ (differences) & $\pi$ (partial sums) \\
\midrule
PPP (1\,118 triples)  & 5.36 & \textbf{2.30} & 1.56 & 1.50 \\
CCC (46\,675 triples) & 5.11 & \textbf{1.66} & 1.60 & 1.84 \\
\midrule
Gap (PPP $-$ CCC) & $+0.25$ & $+0.63$ & $-0.04$ & $-0.34$ \\
\bottomrule
\end{tabular}
\caption{Geometric decomposition of $\rho_3$ at $N = 1025$.}
\label{tab:decomp}
\end{table}

\begin{observation}[Nearly cancelling components]\label{obs:cancel}
  PPP triples have higher $\sigma$ than CCC triples ($+0.63$), because
  $\rad_{\mathrm{odd}}(p) = p$ for primes while
  $\rad_{\mathrm{odd}}(n) \ll n$ for smooth composites.  However, PPP triples
  have \emph{lower} $\pi$ ($-0.34$), because the partial sums $p_i + p_j$ of
  two primes introduce fewer new prime factors than sums involving smooth
  composites.  These two effects nearly cancel, leaving a net PPP--CCC gap of
  only~$0.25$.
\end{observation}

\subsection{Absence of the Band Shifting Law}

\begin{theorem}[No ternary BSL]\label{thm:noBSL}
  The static conduit variable $\xi = 2\log(\rad_{\mathrm{odd}}(N))/\log N$
  has no predictive power for the ternary conductor:
  \begin{equation}
    R^2(\langle\rho_3\rangle_{\mathrm{PPP}} \text{ vs.\ } \xi) = 0.0002.
  \end{equation}
  By contrast, $\langle\rho_3\rangle$ scales with $\log N$:
  \begin{equation}\label{eq:scaling}
    \langle\rho_3\rangle_{\mathrm{PPP}}
    \approx 0.46\,\log N + 2.06,
    \qquad R^2 = 0.977.
  \end{equation}
\end{theorem}

\begin{remark}[Scaling is not a law]\label{rem:scaling}
  The $\log N$ dependence in~\eqref{eq:scaling} is a \emph{statistical scaling
  effect}: larger $N$ means larger summands and differences, which involve more
  distinct primes.  This is fundamentally different from the binary BSL, where
  $\rad_{\mathrm{odd}}(M)$ is a \emph{specific arithmetic quantity} that
  deterministically shifts the conductor band.  The ternary $R^2 = 0.977$
  against $\log N$ reflects a dimensional trend, not an algebraic law.
\end{remark}

Figure~\ref{fig:geometric} presents the corrected conductor landscape.

\begin{figure}[H]
  \centering
  \includegraphics[width=\textwidth]{../figures/fig_geometric.pdf}
  \caption{\textbf{Left:} Density of $\rho_3$ for PPP and CCC triples at
  $N = 1025$, computed from the true discriminant.  The distributions heavily
  overlap. \textbf{Centre:} Geometric decomposition showing that the PPP--CCC
  gap arises from the summand component $\sigma$, partially offset by the
  partial-sum component $\pi$. \textbf{Right:} $\langle\rho_3\rangle$ scales
  with $\log N$ ($R^2 = 0.977$), but shows no correlation with
  $\xi = 2\log(\rad_{\mathrm{odd}}(N))/\log N$.}
  \label{fig:geometric}
\end{figure}

\subsection{Driving variable: differences and partial sums}

\begin{proposition}[Correlation structure]\label{prop:corr}
  For PPP triples at $N = 1025$:
  \begin{align}
    \mathrm{Corr}(\sigma,\, \rho_3) &= 0.51, \\
    \mathrm{Corr}(\delta + \pi,\, \rho_3) &= 0.95.
  \end{align}
  The summand contribution $\sigma$ accounts for only $42.8\%$ of
  $\langle\rho_3\rangle$; the remaining $57.2\%$ comes from difference and
  partial-sum primes.  The total conductor is overwhelmingly determined by how
  many \emph{new} primes appear in the differences $(p_i - p_j)$ and partial
  sums $(N - p_k)$.
\end{proposition}

\subsection{PPP--CCC gap instability}

\begin{proposition}[Unstable floor gap]\label{prop:gap}
  Over odd $N \in [501, 2001]$:
  \begin{itemize}
  \item The floor gap $\rho_{3,\min}^{\mathrm{PPP}} - \rho_{3,\min}^{\mathrm{CCC}}$
    fluctuates between $-0.5$ and $+2.2$, with occasional sign reversals.
  \item The mean gap $\langle\rho_3\rangle_{\mathrm{PPP}} -
    \langle\rho_3\rangle_{\mathrm{CCC}} = 0.10 \pm 0.07$ is positive but
    small and statistically marginal.
  \end{itemize}
\end{proposition}

Figure~\ref{fig:gap} displays the gap stability.

\begin{figure}[H]
  \centering
  \includegraphics[width=0.9\textwidth]{../figures/fig_gap.pdf}
  \caption{PPP--CCC gap across odd $N \in [501, 2001]$.  The floor gap (red)
  is highly volatile with occasional sign reversals.  The mean gap (blue) is
  consistently positive but narrow ($\approx 0.1$).  Compare with the binary
  case, where the floor gap exceeds $2$ units and never changes sign.}
  \label{fig:gap}
\end{figure}


% ═══════════════════════════════════════════════════════════════════════════════
\section{Why Binary Is Special}\label{sec:special}

The entire conductor rigidity framework rests on a single algebraic identity:
\begin{equation}\label{eq:identity}
  p^2 - q^2 = (p - q)(p + q) = (p - q) \cdot N.
\end{equation}
This identity has three consequences that are \emph{unique to genus~2}:

\begin{enumerate}
\item \textbf{$N$ as independent discriminant factor.}
  Equation~\eqref{eq:identity} forces $N \mid (p^2 - q^2)$, so every prime
  $r \mid N$ divides $\Delta_{\mathrm{bin}}$.  This creates a universal
  static conduit: $\rad_{\mathrm{odd}}(N/2)$ is a deterministic,
  $N$-dependent quantity that enters every Goldbach pair's conductor at the
  same weight.

\item \textbf{Conduit--boundary separation.}
  Because the static conduit is a multiplicative factor, Chen's ratio
  decomposes cleanly as $\rho = \underbrace{2\xi}_{\text{conduit}} +
  \underbrace{\beta}_{\text{boundary}}$, where $\xi$ depends only on $N$ and
  $\beta$ depends only on the pair $(p, q)$.  This separation is what makes
  the BSL a \emph{law}: fixing $N$ fixes $\xi$, so $\rho$ varies only through
  the boundary~$\beta$.

\item \textbf{Algebraic vacuum at $N = 2^k$.}
  When $N = 2^k$, the static conduit $\rad_{\mathrm{odd}}(N/2) = 1$ vanishes,
  creating an ``algebraic vacuum''~\cite{Chen2026AV} in which the conductor is
  minimised.  This vacuum is the mechanism behind the ``ground-state
  privilege'' of prime pairs.
\end{enumerate}

In genus~3, \emph{none of these hold}.  The ternary analogue of~\eqref{eq:identity}
is $p_i^2 - p_j^2 = (p_i - p_j)(N - p_k)$, but $N - p_k$ is
\emph{triple-dependent}, not universal.  There is no factorisation that
extracts $N$ as an independent term.  Consequently:

\begin{itemize}
\item No static conduit exists $\Rightarrow$ no BSL ($R^2 = 0.0002$).
\item No conduit--boundary separation $\Rightarrow$ the decomposition
  $\rho_3 = \sigma + \delta + \pi$ has \emph{all} components varying
  simultaneously.
\item No algebraic vacuum $\Rightarrow$ the PPP--CCC gap is small and
  unstable.
\end{itemize}


% ═══════════════════════════════════════════════════════════════════════════════
\section{The Analytic Gap}\label{sec:gap-section}

\begin{remark}[What is and what is not proved]\label{rem:honest}
  \emph{Proved unconditionally:}
  \begin{itemize}
  \item The genus-3 discriminant~\eqref{eq:expanded} and the absence of an
    independent $N$-factor (Proposition~\ref{prop:disc},
    Remark~\ref{rem:critical}).
  \item The geometric decomposition $\rho_3 = \sigma + \delta + \pi$
    (Definition~\ref{def:proxy}).
  \end{itemize}

  \emph{Established computationally ($N \in [501, 4001]$):}
  \begin{itemize}
  \item Absence of ternary BSL: $R^2 = 0.0002$ vs.\ $\xi$
    (Theorem~\ref{thm:noBSL}).
  \item Scaling $\langle\rho_3\rangle \approx 0.46\,\log N + 2.06$
    ($R^2 = 0.977$).
  \item PPP--CCC mean gap: $0.10 \pm 0.07$
    (Proposition~\ref{prop:gap}).
  \item Correlation structure: $\delta + \pi$ drives $95\%$ of PPP variance
    (Proposition~\ref{prop:corr}).
  \end{itemize}

  \emph{Not proved:}
  \begin{itemize}
  \item That no \emph{alternative} static variable controls
    $\langle\rho_3\rangle$ in genus~3.
  \item Any new result on the Goldbach conjecture (binary or ternary).
  \end{itemize}
\end{remark}


% ═══════════════════════════════════════════════════════════════════════════════
\section{Conclusion}

The attempt to extend conductor rigidity from genus~2 to genus~3 fails, and
the failure is illuminating.  The binary BSL is not a statistical pattern
but the geometric shadow of a specific algebraic identity:
$p^2 - q^2 = (p - q)N$.  When this identity breaks---as it must in the
ternary setting, where no analogue extracts $N$ as an independent
factor---the conductor landscape transitions from deterministic rigidity to
statistical scaling.

Table~\ref{tab:summary} summarises the comparison.

\begin{table}[H]
\centering
\begin{tabular}{lcc}
\toprule
Property & Binary ($g = 2$) & Ternary ($g = 3$) \\
\midrule
$N$ in discriminant & Independent factor & Via $(N - p_k)$ only \\
Static conduit & $\rad_{\mathrm{odd}}(M)$ & Does not exist \\
BSL $R^2$ vs.\ $\xi$ & $0.997$ & $0.0002$ \\
$\langle\rho\rangle$ driver & Algebraic (conduit) & Statistical ($\log N$ scaling) \\
PPP--CCC mean gap & $> 0.5$, stable & $0.10 \pm 0.07$, marginal \\
PPP--CCC floor gap & Always positive & Fluctuates, sign reversals \\
Algebraic vacuum & $\rad_{\mathrm{odd}}(M) = 1$ at $N = 2^k$ & None \\
\bottomrule
\end{tabular}
\caption{Binary vs.\ ternary conductor framework: a structural comparison.}
\label{tab:summary}
\end{table}

This paper thus serves two purposes.  First, it precisely delineates the
\emph{applicability boundary} of the conductor rigidity programme: the
framework is native to genus~2 and cannot be directly lifted to higher genera.
Second, by identifying the exact algebraic mechanism that fails ($N$ as
independent discriminant factor), it reinforces the understanding of \emph{why}
binary conductor rigidity works: the identity $p + q = N$ embeds the target
integer into the discriminant with a universality that has no analogue in
ternary or higher-order decompositions.

The binary Goldbach conjecture occupies a privileged position not because it is
the simplest additive problem, but because it is the \emph{only} additive
problem where the target integer enters the Frey discriminant as a universal
geometric invariant.


% ═══════════════════════════════════════════════════════════════════════════════
\section*{Acknowledgments}

The computational scans were performed using \texttt{ternary\_geometric.py},
available at
\url{https://github.com/Ruqing1963/goldbach-ternary-conductor-boundary}.
All results were produced with Python~3.12, NumPy~2.4, and Matplotlib~3.10.
This work builds on~\cite{Chen2026DS,Chen2026AV,Chen2026GMII}.

An earlier version of this paper (circulated as ``Ternary Conductor Rigidity
in $\GSp(6)$'') contained a flawed conductor proxy that artificially injected
$\rad_{\mathrm{odd}}(N)$ into the discriminant.  The author thanks the
reviewer whose analysis of the true genus-3 discriminant led to the corrected
treatment presented here.


% ═══════════════════════════════════════════════════════════════════════════════
\begin{thebibliography}{10}

\bibitem{Chen2026GMII}
R.~Chen, \emph{The Goldbach mirror~{II}: geometric foundations of conductor
rigidity and the static conduit in $\GSp(4)$}, Zenodo, 2026.
\url{https://zenodo.org/records/18719056}

\bibitem{Chen2026AV}
R.~Chen, \emph{The algebraic vacuum: zero-ramification conductor model for the
Goldbach conjecture at $N = 2^k$}, Zenodo, 2026.
\url{https://zenodo.org/records/18720040}

\bibitem{Chen2026DS}
R.~Chen, \emph{Dynamic stability of the Goldbach locus: conductor orbit
propagation and the Band Shifting Law in $\GSp(4)$}, Zenodo, 2026.

\bibitem{Vinogradov1937}
I.\,M.~Vinogradov, \emph{Representation of an odd number as a sum of three
primes}, Dokl.\ Akad.\ Nauk SSSR \textbf{15} (1937), 169--172.

\bibitem{Helfgott2013}
H.\,A.~Helfgott, \emph{The ternary Goldbach conjecture is true},
arXiv:1312.7748, 2013.

\end{thebibliography}

\end{document}
